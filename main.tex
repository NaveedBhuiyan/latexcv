%-------------------------------------------------------------------------------------------------
% LaTeX Resume for Professionals
% Author : Rodrigo Zanini
% License : CC BY 4.0 https://creativecommons.org/licenses/by/4.0/
%-------------------------------------------------------------------------------------------------
\documentclass{article}

\usepackage[T1]{fontenc}
\usepackage[sfdefault,light]{FiraSans}
\usepackage{fontawesome}
\usepackage{xcolor}
\usepackage{calc}

\usepackage[cm,headings]{fullpage}

\usepackage{enumitem}
\usepackage{tabularx}
\usepackage{fancyhdr}
\pagestyle{fancy}

\usepackage{hyperref}
\hypersetup{
    colorlinks=true,
    urlcolor=primary,
}
%-------------------------------------------------------------------------------------------------
%---------- CUSTOM COMMANDS
%-------------------------------------------------------------------------------------------------
%---------- Icons
\newcommand{\icon}[1]{\begin{tabular}{p{\iconWidth}}{#1}\end{tabular}}
\renewcommand{\section}[2]{\vspace{0.5em}\color{primary}\textbf{{#2}{#1}}\hrule\color{black}}

%---------- Head
\renewcommand{\headrulewidth}{0pt}
\newcommand{\cvText}[1]{#1\vspace{0.75em}}

%---------- Entries
% Entry with Name, Time, Position, Location and Details
\newcommand{\cvEntryNTPLD}[5]{
    \begin{itemize}[leftmargin=\entryMargin]
        \item[]
            \begin{tabularx}{\textwidth-\entryMargin}{Xr}
                \textbf{\color{black}#1} & {\color{secondary}\small#2} \\
                \textit{\color{secondary}\small#3} & \textit{\color{secondary}\small#4} \\
            \end{tabularx}\vspace{-0.75em}
        \begin{itemize}[labelsep=\entryMargin-1.5em,leftmargin=\entryMargin]
            \small#5
        \end{itemize}
    \end{itemize}
    \vspace{0.4em} % Adds extra space at the end
}
% Entry with Name, Time, Position and Location
\newcommand{\cvEntryNTPL}[4]{
    \begin{itemize}[leftmargin=\entryMargin]
        \item[]
            \begin{tabularx}{\textwidth-\entryMargin}{Xr}
                \textbf{\color{black}#1} & {\color{secondary}\small#2} \\
                \textit{\color{secondary}\small#3} & \textit{\color{secondary}\small#4} \\
            \end{tabularx}\vspace{-0.75em}
    \end{itemize}
}
% Entry with Name, Time and Details
\newcommand{\cvEntryNTD}[3]{
    \begin{itemize}[leftmargin=\entryMargin]
        \item[]
            \begin{tabularx}{\textwidth-\entryMargin}{Xr}
                \textbf{\color{black}#1} & {\color{secondary}\small#2} \\
            \end{tabularx}\vspace{-0.75em}
        \begin{itemize}[labelsep=\entryMargin-1.5em,leftmargin=\entryMargin]
            \small#3
        \end{itemize}
    \end{itemize}
}
% Entry with Name and Details
\newcommand{\cvEntryND}[1]{
    \small
    \begin{itemize}[leftmargin=\entryMargin]
        #1
    \end{itemize}
    \normalsize
}
% Entry for Languages (up to 4)
\newcommand{\cvItemL}[8]{
    \item[]
        \begin{tabularx}{\textwidth-\entryMargin}{XXXX}
            {\textbf{#1} \textit{#2}}&{\textbf{#3} \textit{#4}}&{\textbf{#5} \textit{#6}}&{\textbf{#7} \textit{#8}}
        \end{tabularx}\vspace{-0.25em}
}
% Entry for Academical Projects
\newcommand{\cvEntryAcademicalNTD}[3]{\cvEntryNTD{\educationIcon{Academical: }#1}{#2}{#3}}
% Entry for Personal Projects
\newcommand{\cvEntryPersonalNTD}[3]{\cvEntryNTD{\personalIcon{Personal: }#1}{#2}{#3}}
% Entry for Work Projects
\newcommand{\cvEntryWorkNTD}[3]{\cvEntryNTD{\workIcon{Work: }#1}{#2}{#3}}

% Bullet point under an entry's details list
\newcommand{\cvItem}[1]{\item[\bulletIcon]{#1\vspace{-0.25em}}}
% Special bullet point under an entry's details list
%\newcommand{\cvItemS}[1]{\item[\accentIcon]{#1\vspace{0.5em}}}
%%%%%%%%%%%%%%%%%
\usepackage{xcolor}
% Define the light gray color
\definecolor{lightgray}{rgb}{0.9, 0.9, 0.9}
% Example command usage
\newcommand{\cvItemS}[1]{\item[\accentIcon]{#1\vspace{-0.25em}}}
%%%%%%%%%%%%%%
% Bulletless point under an entry's details list, with Name and Description 
\newcommand{\cvItemND}[2]{\item[]{\textbf{#1}\hspace{1em}#2}\vspace{-0.25em}}

%-------------------------------------------------------------------------------------------------
%---------- SETTINGS HERE
%-------------------------------------------------------------------------------------------------
%---------- Colours
\definecolor{primary}{HTML}{007BFF}
\definecolor{secondary}{HTML}{3C454D}
\definecolor{info}{HTML}{17A2B8}

%---------- Margins
\raggedbottom
\raggedright
\setlength{\tabcolsep}{0in}
\setlength{\voffset}{-0.5cm}
\setlength{\headheight}{3.5em}
\addtolength{\headsep}{-2em}
\addtolength{\oddsidemargin}{-0.25cm}
\addtolength{\evensidemargin}{-0.25cm}
\addtolength{\headwidth}{0.5cm}
\addtolength{\textwidth}{0.5cm}

%---------- Entries
\def \entryMargin{1em}

%---------- Icons
\def \iconWidth{1.5em}
% Predefined icons, based on FontAwesome.
% see https://ctan.mirror.rafal.ca/fonts/fontawesome/doc/fontawesome.pdf for more options.
\def \linkedinIcon{\icon{\faLinkedin}}
\def \phoneIcon{\icon{\faPhone}}
\def \homeIcon{\icon{\faHome}}
\def \emailIcon{\icon{\faEnvelope}}
\def \githubIcon{\icon{\faGithub}}
\def \websiteIcon{\icon{\faGlobe}}

\def \educationIcon{\icon{\faGraduationCap}}
\def \workIcon{\icon{\faBriefcase}}
\def \projectsIcon{\icon{\faFlask}}
\def \communicationIcon{\icon{\faComments}}
\def \awardsIcon{\icon{\faTrophy}}
\def \skillsIcon{\icon{\faGears}}
\def \interestsIcon{\icon{\faGamepad}}

\def \expertIcon{\icon{\faStar}}
\def \proficientIcon{\icon{\faStarHalfFull}}
\def \noviceIcon{\icon{\faStarO}}
\def \personalIcon{\icon{\faUser}}

\def \bulletIcon{\icon{\faAngleRight}}
\def \accentIcon{\icon{\faAngleDoubleRight}} % \faCaretRight \faAngleDoubleRight \faCode


%-------------------------------------------------------------------------------------------------
%---------- DATA HERE
%-------------------------------------------------------------------------------------------------
%---------- Header data
\def \name{Naveed Bhuiyan}
\def \nameSuffix{}
\def \subtitleText{Automotive Engineer}
\def \summaryText{Something about myself (mini cover letter or summary}

\def \linkedinLink{https://www.linkedin.com/in/naveedbhuiyan/}
\def \linkedinText{/naveedbhuiyan}

\def \phoneText{+49 1774768986}

%\def \homeText{}

\def \emailLink{naveed.bhuiyan@rwth-aachen.de}
\def \emailText{naveed.bhuiyan@rwth-aachen.de}

\def \githubLink{https://github.com/NaveedBhuiyan}
\def \githubText{/NaveedBhuiyan}

%\def \websiteLink{}
%\def \websiteText{}

%---------- Header format
\def \fullName{\textbf{\huge\name}\large\hspace{0.3em}\textit{\nameSuffix}}
\def \subtitle{\textit{\small\cvText{\subtitleText}}}
\def \summary{\cvText{\summaryText}}
\def \linkedin{\small\linkedinIcon\href{\linkedinLink}{\linkedinText}}
\def \phone{\small\phoneIcon{\phoneText}}
%\def \home{\small\homeIcon{\homeText}}
\def \email{\small\emailIcon\href{\emailLink}{\emailText}}
\def \github{\small\githubIcon\href{\githubLink}{\githubText}}
%\def \website{\small\websiteIcon\href{\websiteLink}{\websiteText}}


%-------------------------------------------------------------------------------------------------
%---------- START
%-------------------------------------------------------------------------------------------------
\begin{document}
%-------------------------------------------------------------------------------------------------
%---------- HEADER
%-------------------------------------------------------------------------------------------------
\fancyhead[L]{
    \begin{tabular}[c]{l}
        {\fullName}\\
        {\subtitle}
    \end{tabular}
    \vspace{-0.75em}
}
\fancyhead[R]{
    \begin{tabular}[c]{l@{\hspace{1em}}l@{\hspace{1em}}l}
        % Configure the order in which the header data appears. Must be in 3 colums.
        {\phone} & {\github} & {\email} \\
        {\home} & {\website} & {\linkedin}\\
        \vspace{0.5em}
    \end{tabular}
    \vspace{-0.75em}
}
%-------------------------------------------------------------------------------------------------
%---------- INTRODUCTION
%-------------------------------------------------------------------------------------------------
%\summary\\ % Comment out if not using
%-----------------------------------------------------------------------------------
%-------------------------------------------------------------------------------------------------
%---------- SKILLS
%-------------------------------------------------------------------------------------------------
%\newpage %Uncomment if this section is not entirely on this page
\section{Skills}{\skillsIcon}
\cvEntryND{
    \cvItemND{Programming Languages}{Python, C++}
    \cvItemND{Machine Learning}{Pytorch, Tensorflow}
    \cvItemND{DevOps and Cloud Infrastructure }{Git, CI/CD, AWS, Kubernetes, Docker}
    \cvItemND{Specialization and Other Skills}{ROS,REST API, FLASK, computer vision, Carla, Linux, SQL, Deutsch(B1)}
}
%-------------------------------------------------------------------------------------------------
%---------- WORK EXPERIENCE
%-------------------------------------------------------------------------------------------------
\section{Work Experience}{\workIcon}
\cvEntryNTPLD
    {IAV GmbH}{Nov 2022 -- Present}
    {Developer Engineer}{Dresden}
    {
    \cvItemS{Skills used: \colorbox{lightgray}{\strut Python} \colorbox{lightgray}{\strut Pytorch} \colorbox{lightgray}{\strut Computer Vision} \colorbox{lightgray}{\strut REST API} \colorbox{lightgray}{\strut CI/CD} \colorbox{lightgray}{\strut AWS} \colorbox{lightgray}{\strut Docker} \colorbox{lightgray}{\strut Kubernetes} \colorbox{lightgray}{\strut SQL} \colorbox{lightgray}{\strut Linux}}
    \cvItem{Leveraged the capabilities of PyTorch on sophisticated machine learning techniques such as Generative Adversarial Networks (GANs) to craft style transfer models.}
    \cvItem{Implemented Multi-head Neural Network for tasks including semantic segmentation, lane detection and object detection using Pytorch and extensive knowledge of computer vision.}
    \cvItem{Created seamless CI/CD pipeline for building and pushing Docker images, while automating the deployment process by using jinja2 templates for generating Kubernetes deployment charts and deployment into AWS.}
    \cvItem{Collaborated with an engineering team on a high-impact project to develop a Software in the Loop (SIL) application for our
customers Astemo and Mobileye. Involved in integrating modules into a Python based application and designing robust APIs and comprehensive KPI module to track and analyze performance metrics.}
    \cvItem{Integrated and tested new features into existing Python-based application using Pytest as the unit testing framework, ensuring zero bugs.}
}
\cvEntryNTPLD
    {IAV GmbH}{April 2022 -- Sep 2022}
    {Thesis Student}{Dresden}{
    \cvItemS{Skills used: \colorbox{lightgray}{\strut Python} \colorbox{lightgray}{\strut Pytorch} \colorbox{lightgray}{\strut Computer vision} \colorbox{lightgray}{\strut Carla} \colorbox{lightgray}{\strut Linux}}
    \cvItemS{Master Thesis: "Trajectory Prediction of vehicles within the range of the Ego vehicle, based on predicted Occupancy grid
maps obtained from a Generative adversarial network.”}
    \cvItem{Developed a novel machine learning approach that employs an Autoencoder model as a generator within a Generative Adversarial Network (GAN) for trajectory prediction from Predicted occupancy grid maps using pytorch.}
    \cvItem{Integrated the model into a Software in the Loop (SIL) setup using Carla simulation, enabling dynamic trajectory predictions during the simulation run.}
}
\cvEntryNTPLD
    {IAV GmbH}{May 2021 -- Nov 2021}
    {Internship}{Dresden}{
    \cvItemS{Skills used: \colorbox{lightgray}{\strut Python} \colorbox{lightgray}{\strut Tensorflow} \colorbox{lightgray}{\strut Carla}}
    \cvItem{Designed and developed an LSTM-based machine learning autoencoder model for predicting OGM (occupancy grid maps).}
    \cvItem{Collected point cloud data via a Lidar sensor attached to the simulated vehicle, to create OGMs for each timestamp for test and train dataset.}
    \cvItem{Achieved an accuracy of 82\% Using Tensorflow as primary toolkit and focal loss as the loss function.}    
}
\cvEntryNTPLD
    {Ecurie Aix- Formula Student Team RWTH-Aachen}{May 2020 -- April 2021}
    {Work Student}{Aachen}{
    \cvItemS{Skills used: \colorbox{lightgray}{\strut Python} \colorbox{lightgray}{\strut C++} \colorbox{lightgray}{\strut ROS} \colorbox{lightgray}{\strut Linux}}
    \cvItem{Integrated data fusion of Camera and Lidar sensor data using Python and C++ for better system accuracy.}
    \cvItem{Tested with ROS on Linux platform to enhance software reliability.}
   }
%---------- EDUCATION
%-------------------------------------------------------------------------------------------------
\section{Education}{\educationIcon}
\cvEntryNTPL{RWTH Aachen}{2019 – 2022}{M.Sc Automotive Engineering}{Germany}
\cvEntryNTPL{BUET}{2014 – 2018}{B.Sc Mechanical Engineering}{Bangladesh}
\end{document}